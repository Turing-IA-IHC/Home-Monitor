% ========================= Páginas de presentación =========================
    % PORTADAS
    \portadaTFG		% Portada pral.
    \portadillaTFG	% Portada interior (con tutor(a) y co-tutor(a) si existe).
    
    %\input{Creditos}
    \tribunalTFG % Página para calificaciones del tribunal
    \dedicado{A mi familia y a Jenny \\ % A alguien muy especial
    Por tolerarme durante el desarrollo de este trabajo. \\
    Y a Yova\\ que por su inicial y durante el largo recorrido.} 
    

% ========================= Resumen =========================
    %--- Ajustes del documento.
    \pagestyle{plain}	% Páginas sólo con numeración inferior al pie
    
    %--- Resumen en español
    \selectlanguage{spanish} % Selección de idioma del resumen.
    \cleardoublepage % Para modificar el contador de página antes de añadir bookmark
    \phantomsection  % OJO: Necesario con hyperref
    \pdfbookmark[0]{Resumen}{idx_resumen}% idx_resumen.0 % Bookmark en PDF
    \addcontentsline{toc}{chapter}{Resumen} % Añade al TOC.
    
    \begin{abstract}
    %\textcolor{red}{En una página como máximo, el resumen explicará de modo breve la problemática que trata de resolver el TFG \emph{(el `qué')}, la metodología para  abordar su solución  \emph{(el `cómo')} y los resultados obtenidos.}
    \end{abstract}
    %---
    
    %--- Resumen en inglés
    % Abstract
    \selectlanguage{english} % Selección de idioma del resumen.
    \cleardoublepage
    \phantomsection % OJO: Necesario con hyperref
    \pdfbookmark[0]{Abstract}{idx_abstract}% idx_abstract.0 % Bookmark en PDF
    \begin{abstract}
    
    \end{abstract}
    %---

    %--- Ajuste del idioma para el resto del documento.
    \ifspanish
    	\selectlanguage{spanish}% Emplea idioma español
    \else
    	\selectlanguage{english}% Emplea idioma inglés
    \fi

% ========================= Agradecimientos =========================
    \cleardoublepage
    \phantomsection % OJO: Necesario con hyperref
    \pdfbookmark[0]{Agradecimientos}{idx_agrad}% idx_agrad.0 % Bookmark en PDF
    
    \chapter*{Agradecimientos} % Opción con * para que no aparezca en TOC ni numerada
    \addcontentsline{toc}{chapter}{Agradecimientos} % Añade al TOC.
    
    %Aunque es un apartado opcional, haremos bueno el refrán \emph{<<es de bien nacidos, ser agradecidos>>} si empleamos este espacio como un medio para agradecer a todos los que, de un modo u otro, han hecho posible que el TFG \emph{<<llegue a buen puerto>>}. Esta sección es ideal para agradecer a familiares, directores, profesores, compañeros, amigos, etc. 
     
     %Estos agradecimientos pueden ser tan personales como se desee e incluir anécdotas y chascarrillos, pero nunca deberían ocupar más de una página.
    
    \makeatletter		
    \begin{flushright}
    	\vspace{1,5cm}
    	\textit{\@autor}\\
    	\@lugarDef, \@yearDef
    \end{flushright}
    \makeatother
    
    
% ========================= Notaciones especiales =========================
    %\cleardoublepage
    %\phantomsection % OJO: Necesario con hyperref
    %\pdfbookmark[0]{Notación}{idx_notacion}% idx_notacion.0 % Bookmark en PDF
    %
    %\chapter*{Notación} % Opción con * para que no aparezca en TOC ni numerada
    %\addcontentsline{toc}{chapter}{Notación} % Añade al TOC.
    %
    %Ejemplo de lista con notación (o nomenclatura) empleada en la memoria del %TFG.\footnote{Se incluye unicamente con propósito de ilustración, ya que el %documento no emplea la notación aquí mostrada.}
    %
    %\begin{tabular}{r r p{0.8\linewidth}}
    %$A, B, C, D$	& : & Variables lógicas. \\
    %$f, g, h$		& :	& Funciones lógicas. \\
    %$\cdot$			& : & Producto lógico (AND). A menudo se omitirá como en $A 
    %B$ en lugar de $A \cdot B$.\\
    %$+$				& : & Suma aritmética o lógica (OR) dependiendo del 
    %contexto.\\
    %$\oplus$		& : & OR exclusivo (XOR).\\
    %$\overline{A}$ o ${A}'$	& : & Operador NOT o negación.
    %\end{tabular}
    

% ========================= Indices =========================
    \setindexnames % Ajusta nombres (sólo en español).
    \pagestyle{fancy} % Estilo de página ajustado por fancyhdr
    
    %--- Índice general
    \cleardoublepage
    \phantomsection % OJO: Necesario con hyperref
    \pdfbookmark[0]{Índice general}{idx_toc}% idx_toc.0 % Bookmark en PDF
    \tableofcontents  % Índice general
    
    %--- Índice de figuras
    \cleardoublepage
    \phantomsection % OJO: Necesario con hyperref
    \addcontentsline{toc}{chapter}{\listfigurename}
    %\pdfbookmark[0]{\listfigurename}{idx_lof}% idx_lof.0 % Bookmark en PDF
    \listoffigures    % Índice de figuras (opcional)
    %---
    
    %--- Índice de tablas
    \cleardoublepage
    \phantomsection % OJO: Necesario con hyperref
    \addcontentsline{toc}{chapter}{\listtablename}
    %\pdfbookmark[0]{\listtablename}{idx_lot}% idx_lot.0 % Bookmark en PDF
    \listoftables % Índice de tablas (opcional)
    %---
    
    %--- Índice de listados
    \cleardoublepage
    \phantomsection % OJO: Necesario con hyperref
    \addcontentsline{toc}{chapter}{\lstlistlistingname}
    %\pdfbookmark[0]{\lstlistlistingname}{idx_lol}% idx_lol.0 % Bookmark en PDF
    \lstlistoflistings % Índice de listados creados con listings (opcional)
    %---
    
    %--- Índice de algoritmos
    \cleardoublepage
    \phantomsection % OJO: Necesario con hyperref
    \addcontentsline{toc}{chapter}{\listalgorithmcfname}
    %\pdfbookmark[0]{\listalgorithmcfname}{idx_loa}% idx_loa.0 % Bookmark en PDF
    \listofalgorithms % Índice de algoritmos creados con algortihm2e
    %---
    