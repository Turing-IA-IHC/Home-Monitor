% ========================= Parámetros básicos del documento =========================
    \documentclass[ 		                % Clase del documento
    	11pt,				                % Tamaño de letra
    	a4paper,			                % Tamaño de papel
    	twoside,			                % Impresión a doble cara
    	openright,			                % La apertura de cap. a la dcha.
    	final       		                % Versión final
    ]{book}
    
    % --- Geometría de las páginas del documento ---
    \usepackage[			                % Márgenes del documento
    	top=2.5cm,			                % Margen superior
    	bottom=2.5cm,		                % Margen inferior
    	inner=3.5cm,		                % Margen al interior
    	outer=2cm			                % Margen al exterior
    ]{geometry}
    
    \pagestyle{empty}

    % --- Tipografía ---
    \usepackage{textcomp,marvosym,pifont}   % OPT.: Generación de símbolos 
    \usepackage{ccicons}                    % OPT.: Iconos de licencia Creative Commons
    \usepackage{amsmath,amsthm,amssymb}	    % OPT.: Mejoras cuando hay matemáticas
    \usepackage[tt=false]{libertine}        % Libertine con Old-Style Figures [osf]
    \usepackage[libertine]{newtxmath}       % Times
    \usepackage[T1]{fontenc}                % Codificación de salida    
    \usepackage{microtype}	                % Mejoras de microtipografía en el PDF
    
    %--- OPT.: Paquete para incluir menús, paths y teclas de modo "elegante"
    %\usepackage[os=win,hyperrefcolorlinks]{menukeys} 
    %\renewmenumacro{\menu}[>]{menus}                        % OPT.: default:menus
    %\renewmenumacro{\directory}[/]{pathswithblackfolder}    % OPT.: default:paths
    %\renewmenumacro{\keys}[+]{shadowedroundedkeys}          % OPT.: default:roundedkeys

    % --- Idioma ---
    \usepackage[utf8]{inputenx}             % Codificación de entrada
    \usepackage[english,spanish,es-tabla,es-noindentfirst]{babel} % Internacionalización

    % --- Código Fuente ---
    \usepackage{listings}
    \renewcommand\lstlistingname{Algorithm}
    \renewcommand\lstlistlistingname{Algorithms}
    \def\lstlistingautorefname{Alg.}
    
    \usepackage[spanish]{uclmTFGesi}

% ========================= Configuración de elementos =========================
    % --- Gráficos ---
    \usepackage{graphicx}	                    % Inclusión de figuras
    \usepackage{subcaption}	                    % OPT.: Inclusión de subfiguras
    \graphicspath{{./imgs/}}                    % Path de búsqueda de ficheros gráficos
    \DeclareGraphicsExtensions{.pdf,.png,.jpg}  % Precedencia de extensiones

    % --- Tablas ---    
    \usepackage{tabularx,booktabs}	            % OPT.: Ajustes para tablas
    %\captionsetup[table]{skip=4pt} 	            % Separación del caption en las tablas
    \usepackage{longtable}

    % --- Títulos de figuras y tablas ---
    \usepackage[%
    	margin=10pt,		                    % Margen
    	font=small,			                    % Tamaño de tipografía
    	labelfont=bf,		                    % Prefijo-Etiqueta en negrita
    	format=hang			                    % Formato
    ]{caption}

% ========================= Configuración de Bibliografía =========================
    %--- Bibliografía: Biblatex con biber.
    \usepackage[
    	backend=biber, 		% Backend
    	sortcites,
    	defernumbers=true, 	% Para numerar al final
    	style=numeric-comp, % Estilo numérico condensado
    	% Descomentar las opciones siguientes para bibliografía multilingüe
        % autolang=other, 	% Requerido para opción multilingüe
        % language=auto   	% Requerido para opción multilingüe
    ]{biblatex}

    % Línea añadida para eliminar el idioma de la fuente bibliográfica.
    \AtEveryBibitem{\clearfield{note} \clearlist{language}}
    \addbibresource{bibliography.bib} 	% Fichero de bibliografía.


% ========================= Formatos Indices =========================
    \usepackage{makeidx} % OPT.: Indice temático
    \makeindex           % OPT.: Procesamiento de índice temático


